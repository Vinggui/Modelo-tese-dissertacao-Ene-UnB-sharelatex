\titulolinhas{}

%% [sub-título]{Título original}
\titulolinhas[Versão em LaTeX - Overleaf V2]{Template em LaTeX para dissertações do ENE:}

\maketitle

%% No caso de haver outros autores (máximo 3), crie mais um "\autores"
\autores{Margaret Hamilton}
\autores{Minch Yoda}

%% {Grau desejado em detalhes}{Tipo de monografia}{Departamento}{Grau desejado}{Sigla do departamento}{Programa do aluno ou departamento novamente]}
\grau{Doutor Engenheiro em Sistemas Eletrônicos e Automação}{Tese de doutorado}{Engenharia El\'{e}trica}{Doutor}{ENE}{Engenharia de Sistema Eletrônicos e Automação}

\datainfo{Janeiro}{01}{2017}

%% {Traduzido para outro língua}{Título original}
\titulolinguas{LaTeX Template - Overleaf v2 Version}{Template em Latex para dissertações do ENE: Versão Lyx}

%% Senão houver co-orientador, deixe o campo [] vazio.
\orientador[Who, Dr.]{Strange, Dr.}

%% Para adicionar mais membros da banca: Após o último membro, crie uma nova linha (enter) e vá na aba superior esquerda (abaixo de “file”) e selecione “membro da banca”. Para deletar membros, apenas delete o item correspondente
\membrodabanca[Orientador]{Prof. Dr. Strange \textendash{} ENE/Universidade de Brasília }

\membrodabanca[Membro Interno]{Prof. Dr. Jhon H. Watson \textendash{} Dep./Universidade }

\membrodabanca[Membro Externo]{Dr. Evil \textendash{} Dep./Universidade }

\catalogonome[Hamilton, M., Yoda, M.]{Hamilton, Margaret; Yoda, Minch,}

%% {Publicação N#}{Palavra chave #1}{chave #2}{chave #3}{chave #4}{n# páginas em romano}{n# páginas final}
\catalogoinfo{PPGEA.TD-001/11}{Palavra 1}{Palavra 2}{Palavra 3}{Palavra 4}{xi}{100}
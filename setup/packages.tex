\usepackage[utf8]{inputenc}
\setcounter{secnumdepth}{3}
\setcounter{tocdepth}{3}
\usepackage{array}
\usepackage{verbatim}
\usepackage{amsmath}
\usepackage{amsthm}
\usepackage{xargs}[2008/03/08]
\usepackage{import}
\usepackage{siunitx}
\usepackage{multirow}
\usepackage[section]{placeins}
\usepackage{url}
\usepackage{breakurl}
%\usepackage[justification=centering]{caption} % Center all captions...

% Used for Code  blocks, comment if not used...
\usepackage{listings}

% Do not change the sequence of this package import.
\usepackage[hidelinks]{hyperref}

% Symbols, acronyms, and notation
% Remove the nonumberlist to insert the pag in which they apear
\usepackage[acronym,nonumberlist,automake,style=tree]{glossaries}
    \providecommand{\notationname}{NOTATION}
    \providecommand{\symbolsabrevname}{LIST OF SYMBOLS AND ABBREVIATIONS}
    \newglossary[slg]{symbolslist}{syi}{syg}{\symbolsabrevname}
    \newglossary[nlg]{notation}{not}{ntn}{\notationname}

\makeatletter
\makeglossaries

%%%%%%%%%%%%%%%%%%%%%%%%%%%%%% LyX specific LaTeX commands.
\providecommand{\LyX}{L\kern-.1667em\lower.25em\hbox{Y}\kern-.125emX\@}
%% Because html converters don't know tabularnewline
\providecommand{\tabularnewline}{\\}

%%%%%%%%%%%%%%%%%%%%%%%%%%%%%% Textclass specific LaTeX commands.
    
 	\usepackage{ft4unb}
  \theoremstyle{plain}
  \ifx\thechapter\undefined
    \newtheorem{thm}{\protect\theoremname}
  \else
    \newtheorem{thm}{\protect\theoremname}[chapter]
  \fi
  \theoremstyle{definition}
  \ifx\thechapter\undefined
    \newtheorem{example}{\protect\examplename}
  \else
    \newtheorem{example}{\protect\examplename}[chapter]
  \fi
  \theoremstyle{definition}
  \ifx\thechapter\undefined
    \newtheorem{defn}{\protect\definitionname}
  \else
    \newtheorem{defn}{\protect\definitionname}[chapter]
  \fi
  \theoremstyle{plain}
  \ifx\thechapter\undefined
    \newtheorem{lem}{\protect\lemmaname}
  \else
    \newtheorem{lem}{\protect\lemmaname}[chapter]
  \fi

%%%%%%%%%%%%%%%%%%%%%%%%%%%%%% User specified LaTeX commands.
%%% PACOTES: FONTE  (VERIFICAR SE ESTÃO INSTALADOS)
\usepackage[T1]{fontenc}


%%% PACOTES: GRAFICOS, TABELAS ETC
\usepackage{graphicx}
  % declare the path(s) where your graphic files are
  \graphicspath{{./figs/}}
  % and their extensions so you won't have to specify these with
  % every instance of \includegraphics
  \DeclareGraphicsExtensions{.pdf,.jpeg,.png,.jpg,.bmp}
\usepackage{multicol}
\usepackage{subfig}
\usepackage{array}

%%% PACOTES: ALGORITMOS
\usepackage{algorithm}
\usepackage{algpseudocode}

%%% PACOTES: OUTROS
\usepackage{pdfpages}


%%% PACOTES NECESSÁRIOS PELO TEMPLATE (JÁ INSTALADOS):
%%% color, [usenames,dvipsnames,svgnames,table]{xcolor}
%%% {eso-pic,graphicx}, mdframed, tcolorbox
%%% times, geometry, import, ifthen, calc,{xstring,xifthen}


% Do not change the sequence of this package import. 
% Also, try NOT TO USE THIS PACKAGE. Just comment and define one of the %\bibliographystyle{}
\usepackage[num]{abntex2cite}
% relative to abntex2cite cite mode
\citebrackets[]
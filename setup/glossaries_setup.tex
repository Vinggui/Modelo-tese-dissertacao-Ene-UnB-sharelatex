% Comment this as you don`t use it
%-----------------------------------------------
% Use os simbolos, acronimos e notacoes com as funcoes a seguir
% To use this acronyms list, check the file glossaries-list/symbos.tex.
%---------------------------------------
% List of symbols definition, do not change it!
%---------------------------------------
\newglossarystyle{symbolsstyle}{%
	\setglossarystyle{index}% base this style on the list style
	\renewcommand{\glsgroupskip}{}% make nothing happen between groups
	\renewcommand*{\glspostdescription}{} 
}

\newglossaryentry{greekletters}{name={Greek symbols},description={\glspar},sort=a}
\newglossaryentry{latinletters}{name={Latin symbols},description={\glspar},sort=b}
%---------------------------------------

% Example for Symbols (USE A TAG "symb:" for symbols):
% \newglossaryentry{symb:pi}
% {
%     type=symbolslist,
%     parent=greekletters,
%     sort=a,
%     symbol={\ensuremath{\pi}}
%     name=pi,
%     description={Ratio of circumference of circle to its diameter}
% }

\newglossaryentry{symb:tot}
{
	type=symbolslist,
	name={\ensuremath{Tot_{time}}},
	description={Total time used}
}

\newglossaryentry{symb:E}
{
	type=symbolslist,
	parent=latinletters,
	name=E,
	description={Total energy consumed}
}

\newglossaryentry{symb:P}
{
	type=symbolslist,
	parent=latinletters,
	name=P,
	description={Average power consumption}
}

\newglossaryentry{symb:pi}
{
	type=symbolslist,
	parent=greekletters,
	sort=a,
	name={\ensuremath{\pi}},
	description={Ratio of circumference of circle to its diameter}
}

% In the text:
%     use \gls{label}, for an first time extended name
% In the text , for plural:
%     \glspl{duck}
% In the text , for Caps and/or plural:
%     \Gls{duck} => Duck
%     \Glspl{duck} => Ducks




 


% To use this acronyms list, check the file glossaries-list/acronyms.tex.
% Example for Glossary:
% \newglossaryentry{latex}
% {
%     name=latex,
%     description={Is a mark up language specially suited 
%     for scientific documents}
% }
%
% To include acronyms into the list of symbols, do as follow:
% * add [parent=acronymslists] after every "\newacronym";
% * Comment the line "\listadeacronimos" in file "thesis.tex";
% * Remove the comment the following line:
%\newglossaryentry{acronymslists}{name={Acronyms list},description={\glspar},sort=c}


\newglossaryentry{maths}
{
    name=mathematic,
    description={Mathematics is what mathematicians do}
}

% Example for Acronyms:
% \newacronym{label}{Acronym}{Full name}
\newacronym{uti}{UTI}{Unidade de Tratamento Intensivo}
\newacronym{scuba}{SCUBA}{Self-Contained Underwater Breathing Apparatus}
\newacronym{radar}{RADAR}{RAdio Detection And Ranging}

% In the text:
%     use \gls{label}, for an first time extend name, plus acronyms
%     use \acrshort{label}, for an acronym only
%     use \acrlong{label}, for a full name only
% In the text , for plural:
%     \glspl{duck}
% In the text , for Caps and/or plural:
%     \Gls{duck}
%     \Glspl{duck}




% To use this acronyms list, check the file glossaries-list/notation.tex.
% Example for Notation (USE A TAG "not:" for notations):
% \newglossaryentry{not:latex}
% {
%     type=notation,
%     name=Latex,
%     text={latex},% This can be used for different Caps in text
%     description={Is a mark up language specially suited 
%     for scientific documents
%     }
% }
%
% In the text:
%     use \gls{label}, for an first time extended name
% In the text , for plural:
%     \glspl{duck}
% In the text , for Caps and/or plural:
%     \Gls{duck} => Duck
%     \Glspl{duck} => Ducks
%
%
% To just add a Symbol without using it, do:
% \glsaddall
% \glsadd{label}

\newglossaryentry{not:vet}
{
	type=notation,
	name=Vet,
	text={vet},
	description={Neste trabalho vetores são representados por letras minúsculas em
		negrito. Matrizes são representadas por letras maiúsculas em negrito.
		Já espaços e conjuntos em geral são representados por letras maiúsculas
		caligráficas.}
}





 

%-----------------------------------------------

%---------------------------------------
% List of symbols definition, do not change it!
%---------------------------------------
\newglossarystyle{symbolsstyle}{%
% 	\setglossarystyle{index}% base this style on the list style
	\renewcommand{\glsgroupskip}{}% make nothing happen between groups
    \setglossarystyle{long3col}% base this style on the list style
    \renewenvironment{theglossary}%
    {\begin{longtable}{>{\raggedright}p{.10\textwidth}p{.75\textwidth}>{\raggedright}p{.20\textwidth}}}%
    {\end{longtable}}%
    
    % If you need the header for this table...
    % \renewcommand*{\glossaryheader}{%  Change the table header
    %   \bfseries Sign & \bfseries Description & \bfseries Unit \\
    %   \hline
    %   \endhead}
    
    \renewcommand*{\glossentry}[2]{%  Change the displayed items
        \mbox{\glstarget{##1}{\glossentryname{##1}}}
        & & \tabularnewline
    }
    \renewcommand{\subglossentry}[3]{%  Change the displayed items
        \glstarget{##2}{\glossentryname{##2}} %
        & \Glossentrydesc{##2}% Description
        & \ifthenelse{\equal{\glsentryunit{##2}}{}}{}{[\glsentryunit{##2}]}  \tabularnewline
    }
}


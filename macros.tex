\global\long\def\mymatrix#1{\boldsymbol{#1}}

\global\long\def\myvec#1{\boldsymbol{#1}}

\global\long\def\mapvec#1{\boldsymbol{#1}}

\global\long\def\dotproduct#1{\langle#1\rangle}

\global\long\def\norm#1{\left\Vert #1\right\Vert }

\global\long\def\sgn{\operatorname{sgn}}

\global\long\def\diag#1{\operatorname{diag}\left(#1\right)}

\global\long\def\diagterm#1{\mathfrak{D}\negmedspace\left\{  \negthinspace#1\negthinspace\right\}  }

\global\long\def\trace#1{\operatorname{tr}\left(#1\right)}
\begin{comment}
Trace of matrix \#1
\end{comment}

\global\long\def\inner#1#2{\langle#1,#2\rangle}
\begin{comment}
Operation: inner product (dot product)
\end{comment}

\global\long\def\outer#1#2{#1\times#2}
\begin{comment}
Operation: Outer product (cross product)
\end{comment}

\global\long\def\skew#1{{\left\{  #1\right\}  ^{\times}} }
\begin{comment}
Skew Symmetric Matrix
\end{comment}

\global\long\def\skewsymproduct#1{\ensuremath{\left\lfloor #1\right\rfloor _{\times}}}
\begin{comment}
Skew-symmetric matrix associated to cross-product with vector \#1
\end{comment}

\global\long\def\imi{\hat{\imath}}

\global\long\def\imj{\hat{\jmath}}

\global\long\def\imk{\hat{k}}

\global\long\def\imvec{\boldsymbol{\imath_{m}}}

\global\long\def\dq#1{\underline{\boldsymbol{#1}}}

\global\long\def\quat#1{\boldsymbol{#1}}

\global\long\def\dualvector#1{\underline{\boldsymbol{#1}}}

\global\long\def\dual{\varepsilon}

\global\long\def\mydual#1{\underline{#1}}

\global\long\def\hamilton#1#2{\overset{#1}{\operatorname{\mymatrix H}}\left(#2\right)}

\global\long\def\hamifour#1#2{\overset{#1}{\operatorname{\mymatrix H}}_{4}\left(#2\right)}

\global\long\def\hami#1{\overset{#1}{\operatorname{\mymatrix H}}}

\global\long\def\tplus{\dq{{\cal T}}}

\global\long\def\getp#1{\operatorname{\mathcal{P}}\left(#1\right)}

\global\long\def\getd#1{\operatorname{\mathcal{D}}\left(#1\right)}

\global\long\def\swap#1{\text{swap}\{#1\}}

\global\long\def\real#1{\operatorname{\mathrm{Re}}\left(#1\right)}

\global\long\def\imag#1{\operatorname{\mathrm{Im}}\left(#1\right)}

\global\long\def\vector{\operatorname{vec}}

\global\long\def\vectorinv{\underline{\operatorname{vec}}}

\global\long\def\mathpzc#1{\fontmathpzc{#1}}

\global\long\def\cost#1#2{\underset{\text{#2}}{\operatorname{\text{cost}}}\left(\ensuremath{#1}\right)}

\global\long\def\dualquaternion#1{\underline{\boldsymbol{#1}}}

\global\long\def\quaternion#1{\boldsymbol{#1}}

\global\long\def\sphereset{\mathcal{S}}

\global\long\def\closedballset{\ensuremath{\mathbb{B}}}
\begin{comment}
The closed unit ball in the Euclidean norm of convenient dimension
\end{comment}

\global\long\def\rationalset{\ensuremath{\mathbb{Q}}}
\begin{comment}
The set of rational numbers
\end{comment}

\global\long\def\integerset{\ensuremath{\mathbb{Z}}}
\begin{comment}
The set of integer numbers
\end{comment}

\global\long\def\naturalset{\ensuremath{\mathbb{N}}}
\begin{comment}
The set of natural numbers
\end{comment}

\global\long\def\realset{\ensuremath{\mathbb{R}}}
\begin{comment}
The set of real numbers
\end{comment}

\global\long\def\complexset{\ensuremath{\mathbb{C}}}
\begin{comment}
The set of complex numbers
\end{comment}

\global\long\def\dualset{\mathbb{D}}
 %
\begin{comment}
The set of dual numbers
\end{comment}
{} 

\global\long\def\quatset{\ensuremath{\mathbb{H}}}
 %
\begin{comment}
The set of quaternions
\end{comment}

\global\long\def\purequatset{\ensuremath{\mathbb{H}_{0}}}
 %
\begin{comment}
The set of pure quaternions
\end{comment}

\global\long\def\unitquatset{\mathcal{S}^{3}}
 %
\begin{comment}
The set of quaternions with unit norm, i.e. the unit 3-sphere
\end{comment}

\global\long\def\dualquatset{\mathbb{H}\otimes\mathbb{D}}
 %
\begin{comment}
The algebra of dual quaternions
\end{comment}
{} %
\begin{comment}
Selig's notation
\end{comment}

\global\long\def\puredualquatset{\ensuremath{\mathbb{D}_{V}}}
 %
\begin{comment}
The set of pure dual quaternions\textemdash dual vectors
\end{comment}

\global\long\def\unitdualquatset{\dq S}
\begin{comment}
The set of dual quaternions with unit norm, i.e. the Study set
\end{comment}
{} 

\global\long\def\StudySet{\mathcal{S}^{3}\otimes\mathbb{D}}
 %
\begin{comment}
The set of dual quaternions with unit norm
\end{comment}
{} %
\begin{comment}
Podemos usar também $\mathcal{S}_{3}\rtimes\mathbb{R}^{3}$
\end{comment}

\begin{comment}
\global\long\def\unitdualquatset{\mathcal{S}^{3}\otimes\mathbb{D}}
 
\end{comment}
\begin{comment}
The set of unit dual quaternions
\end{comment}

\newcommandx\projspace[2][usedefault, addprefix=\global, 1=n]{\mathbb{P}^{#1}\left(#2\right)}
\begin{comment}
Projective space over the field \#1
\end{comment}

\newcommandx\GL[2][usedefault, addprefix=\global, 1=\realset]{\ensuremath{GL\left(#2,#1\right)}}
\begin{comment}
The \#1 dimensional general linear group over the field \#2
\end{comment}

\newcommandx\SL[2][usedefault, addprefix=\global, 1=\realset]{\ensuremath{SL\left(#2,#1\right)}}
\begin{comment}
The \#1 dimensional special linear group over the field \#2
\end{comment}

\global\long\def\SE#1{\ensuremath{SE\left(#1\right)}}
\begin{comment}
The \#1 dimensional special Euclidean group
\end{comment}

\global\long\def\SO#1{\ensuremath{SO\left(#1\right)}}
\begin{comment}
The \#1 dimensional special orthogonal group
\end{comment}

\global\long\def\SU#1{\ensuremath{SU\left(#1\right)}}
\begin{comment}
The \#1 dimensional special unitary group
\end{comment}

\global\long\def\unitquatgroup{\ensuremath{\text{Spin}(3)}}
 %
\begin{comment}
The algebra of quaternions
\end{comment}

\global\long\def\unitdualquatgroup{\text{Spin}(3)\ltimes\mathbb{R}^{3}}
\begin{comment}
The set of unit dual quaternions
\end{comment}

\newcommandx\liealgebraGL[2][usedefault, addprefix=\global, 1=\realset]{\ensuremath{\mathfrak{gl}\left(#2,#1\right)}}
\begin{comment}
The Lie algebra associated to the \#1 dimensional general linear group
over the field \#2
\end{comment}

\newcommandx\liealgebraSL[2][usedefault, addprefix=\global, 1=\realset]{\ensuremath{\mathfrak{sl}\left(#2,#1\right)}}
\begin{comment}
The Lie algebra associated to the \#1 dimensional special linear group
over the field \#2
\end{comment}

\global\long\def\liealgebraSE#1{\ensuremath{\mathfrak{se}\left(#1\right)}}
\begin{comment}
The Lie algebra associated to the \#1 dimensional special Euclidean
group
\end{comment}

\global\long\def\liealgebraSO#1{\ensuremath{\mathfrak{so}\left(#1\right)}}
\begin{comment}
The Lie algebra associated to the \#1 dimensional special orthogonal
group
\end{comment}

\global\long\def\liealgebraSU#1{\ensuremath{\mathfrak{su}\left(#1\right)}}
\begin{comment}
The Lie algebra associated to the \#1 dimensional special unitary
group
\end{comment}

\global\long\def\isomorphism#1#2#3{\ensuremath{#1\overset{#2}{\leftrightarrow}#3}}

\global\long\def\monomorphism#1#2#3{\ensuremath{#1\overset{#2}{\rightarrowtail}#3}}

\global\long\def\epimorphism#1#2#3{\ensuremath{#1\overset{#2}{\twoheadrightarrow}#3}}

\global\long\def\embedding#1#2{\ensuremath{#1\hookrightarrow#2}}

\global\long\def\subgroup#1#2{\ensuremath{#1\le#2}}

\global\long\def\isogroups#1#2{\ensuremath{#1\simeq#2}}

\global\long\def\normalsubgroup#1#2{\ensuremath{#1\triangleleft#2}}

\global\long\def\quotientgroup#1#2{\ensuremath{#1/#2}}

\global\long\def\pointcoordinate#1#2{\ensuremath{#1_{#2}}}
\begin{comment}
Coordinate of point \#1 relative to frame \#2 \textendash{} Note that
Adorno uses the alternative notation \#1\textasciicircum{}\#2
\end{comment}

\global\long\def\frametransform#1#2#3{\ensuremath{#1_{#2#3}}}
\begin{comment}
Transform \#1 which changes coordinates relative to \#2 (frame \#2)
to coordinates relative to \#3 (frame \#3) \textendash{} Note that
Adorno uses \#1\textasciicircum{}\{\#2\}\_\{\#3\}
\end{comment}

\global\long\def\homogeneous#1{\bar{#1}}

\global\long\def\axisangle#1#2{\ensuremath{R\left(\myvec{#1},#2\right)}}
\begin{comment}
Axis-angle representation with axis \#1 and angle \#2
\end{comment}

\global\long\def\quatbilin#1#2{\left[\left[#1,#2\right]\right]}
\begin{comment}
Altmann's notation to quaternion with real part \#1 and vector part
\#2
\end{comment}

\begin{comment}
\global\long\def\quatbilin#1#2{\ensuremath{\left\llbracket #1,#2\right\rrbracket }}
\end{comment}
\begin{comment}
Prettier notation, but need to add \textbackslash{}usepackage\{stmaryd\}
for every subdocument preamble
\end{comment}

